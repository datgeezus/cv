\documentclass[margin,line]{res}
\usepackage{fontawesome}
\usepackage[usenames,dvipsnames,svgnames,table]{xcolor}
\usepackage{hyperref}
\usepackage{url}
\hypersetup{
    colorlinks=true,
    linkcolor=WildStrawberry,
    filecolor=magenta,
    urlcolor=WildStrawberry,
}
\usepackage[para]{footmisc}

\oddsidemargin -.5in
\evensidemargin -.5in
\textwidth=6.0in
\itemsep=0in
\parsep=0in
% if using pdflatex:
%\setlength{\pdfpagewidth}{\paperwidth}
%\setlength{\pdfpageheight}{\paperheight} 

\newenvironment{list1}{
  \begin{list}{\ding{113}}{%
      \setlength{\itemsep}{0in}
      \setlength{\parsep}{0in} \setlength{\parskip}{0in}
      \setlength{\topsep}{0in} \setlength{\partopsep}{0in} 
      \setlength{\leftmargin}{0.17in}}}{\end{list}}
\newenvironment{list2}{
  \begin{list}{$\bullet$}{%
      \setlength{\itemsep}{0in}
      \setlength{\parsep}{0in} \setlength{\parskip}{0in}
      \setlength{\topsep}{0in} \setlength{\partopsep}{0in} 
      \setlength{\leftmargin}{0.2in}}}{\end{list}}


\begin{document}

\name{Jesus Sanchez Sevilla \vspace*{.1in}}

\begin{resume}
\section{\sc Contact}
\vspace{.05in}
\begin{tabular}{l l l}
    \faPhone \hspace{2pt} +52(222)458-7212 & \faLinkedin \hspace{2pt} \href{https://mx.linkedin.com/in/jesus-sanchez-83616221}{Jesus Sanchez} & 15 sur 8322\\
    \faEnvelope \hspace{2pt} \href{mailto:jesus.sanchez@outlook.com}{jesus.sanchez@outlook.com} & \faGithub \hspace{2pt} \href{https://github.com/JeshuaSan}{JeshuaSan} &  Puebla, PUE. 72468, Mexico \\
\end{tabular}


\section{\sc Professional Experience}
{\bf General Electric} Aviation Avionics and Digital Systems, Queretaro, Mexico

\vspace{-.3cm}
{\em Lead Software Engineer} \hfill \\
Technical leader for Mexico's team of an IR\&D project that focuses on 
next-generation flight decks, improving the graphical capabilities of display apps and 
exploring new ways of human-machine interaction.  As a software engineer, I've helped in the design
and implementation of a custom graphics API based on OpenGL ES; designed and implemented
a code generator for translating XML-formatted files into OpenGL C calls,
and a code generator for translating JSON files to
ZMQ\footnote{\href{http://zeromq.org}{http://zeromq.org}} C calls;
developed several display apps, both graphical look and behavior, led the
development of a digital moving map, designed and build an API for remote drawing,
interaction and control.

Ported the graphical API to Linux using the Autotools toolchain and
configured a custom embedded Linux build that included our custom graphics
back-end, ZMQ, example applications of our HMIs and remote debugging through console and
using an IDE.

Outside my main team, I led the design and development of avionics applications,
test metodology and integration support at the client's facilities in USA.
As a part of a larger team, I helped developed and test avionics applications for
one of GE's most important clients, implementing embedded systems design best
practices and guidelines and supported junior team members.

I've conducted training sessions for GE's global sites: USA (MI, CA, FL), UK, China and Mexico;
participated in several presentations showing our work to higher level
executives and co-workers. As the team leader, I keep a close communication with
GE USA to help concrete ideas, as well as define and assign task for the team;
bring new team members get up to speed with the technologies and concepts we use
and support existing team members leveraging my product experience.
Participated in the organization of two successful {\em hackathons} directed towards
both students and professionals wanting to experience GE's engineering challenges.

\section{\sc Education}
{\bf Master of Science in Electronics} focus on embedded systems \\
\vspace*{-.1in}
\begin{list1}
\item[] Benemérita Universidad Autónoma de Puebla, Mexico \hfill{\bf January, 2011 - July 2013}
\begin{list2}
\item Thesis Topic: ``Mobile robot with a computer vision based autonomous navigation''
\end{list2}
\end{list1}

{\bf Bacherol of Science in Mechatronics}   \\
\vspace*{-.1in}
\begin{list1}
\item[] Benemérita Universidad Autónoma de Puebla, Mexico \hfill{\bf August, 2005 - May 2010}
\end{list1}

\section{\sc Patent} 
{\bf MX/I/2018/100659} "Robot móvil para investigación" \footnote{ \href{https://drive.google.com/open?id=1DsnaAByaBixFv9GfyCH6JUAA95fj8aTY}{MX/I/2018/100659}}

\section{\sc Computer Skills} 
\begin{list1}
\item[] Programming Languages:
\begin{list2}
\item C:
    Graphical applications using OpenGL, data structures, OOP concepts and embedded systems best practices;
    embedded systems applications\footnote{\faGithub \href{https://github.com/JeshuaSan/mobile_robot}{\texttt{JeshuaSan/mobile\_robot}}};
    peripheral device libraries\footnote{\faGithub \href{https://github.com/JeshuaSan/dspic33f_pic24h_corelibs}{\texttt{JeshuaSan/dspic33f\_pic24h\_corelibs}}}.
\item Python: Data model processing and code generators; general purpose scripts
\item MATLAB:
    Face recognition\footnote{\faGithub\href{https://github.com/JeshuaSan/matlab\_eigenfaces}{\texttt{JeshuaSan/matlab\_eigenfaces}}},
    image processing\footnote{\faGithub\href{https://github.com/JeshuaSan/MATLAB}{\texttt{JeshuaSan/MATLAB}}}
\item Processing:
    Computer-microcontroller interface\footnote{\faGithub\href{https://github.com/JeshuaSan/processing_oled}{\texttt{JeshuaSan/processing\_oled}}}
\end{list2}
\item[] Software Tools: Visual Studio, Eclipse, Git, Vim, GDB, EagleCAD, Blend, YOCTO\footnote{\href{https://www.yoctoproject.org/}
{https://www.yoctoproject.org/}} project.
\item[] Operating Systems:  Unix/Linux, Windows.
\end{list1}


\end{resume}
\end{document}

